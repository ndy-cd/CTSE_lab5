\documentclass[a4paper]{article}

%\usepackage{fullpage}
\usepackage[14pt]{extsizes}			% размер шрифта
\usepackage[T2A]{fontenc}			% кодировка
\usepackage[utf8]{inputenc}			% кодировка исходного текста
\usepackage[english,russian]{babel}	% локализация и переносы
\usepackage{mathtools}				% математический пакет (включает ams)
\usepackage[thinc]{esdiff}			% производные
\usepackage{graphicx}				% изображения
\usepackage{listings}				% програмный код
\lstset{
	numbers=left               
}
\usepackage{setspace}
\onehalfspacing						% полуторный интервал

\usepackage{tocloft}				% точки в содержании для \section
\renewcommand{\cftsecleader}{\cftdotfill{\cftdotsep}}

\author{Бартая Нодари ФМ-101}
\title{ОТЧЁТ\\ Компьютерные технологии в\\ науке и образовании\\ Задание № 5}
\date{\today}

\begin{document}

\begin{titlepage}
	\begin{center}
	Федеральное государственное бюджетное образовательноe учреждение высшего образования \\
	\textbf{«Челябинский государственный университет» \\ (ФГБОУ ВО «ЧелГУ»)}
	\end{center}
	\begin{center}
		Факультет физический			\\
		Кафедра теоретической физики	\\
		Направление 03.04.02 Физика		\\
		Направленность Теоретическая и математическая физика
	\end{center}
	\vfill
	\begin{center}
		\textbf{ОТЧЁТ}
	\end{center}
	\begin{center}
		\textsc{\textbf{Компьютерные технологии\\ в науке и образовании\\ Задание № 5}}
	\end{center}
	\vfill
	\newbox{\lbox}
	\savebox{\lbox}{\hbox{ПупИванович}}
	\newlength{\maxl}
	\setlength{\maxl}{\wd\lbox}
	\hfill\parbox{10cm}{
		\begin{flushright}
		\hfill Преподаватель:   \hbox to\maxl{Хайбрахманов С. А.\hfill}\\
		\hfill Студент:   \hbox to\maxl{Бартая Н.В.\hfill}\\
		\hfill Группа:   \hbox to\maxl{ФМ - 101\hfill}\\
		\end{flushright}
	}
	\vfill
	\begin{center}
		Челябинск, 2020	
	\end{center}
\end{titlepage}

%\maketitle
\tableofcontents
\setcounter{page}{2}
\newpage
\section{Постановка задачи}
Требуется рассчитать траекторию движения электрона в скрещенных
постоянных магнитном и электрическом полях. 
С компонентами $\mathbf{E}=(E_x, 0, 0)$, $\mathbf{B}=(0, 0, B_z)$ и напряженностями 10 А/м и 10 В/м  соответственно. 
Начальная скорость электрона $\mathbf{v}_0$ направлена перпендикулярно вектору напряженности магнитного поля и составляет 10 см/с.
Геометрия задачи представлена на рисунке ниже.%\ref{problem} (стр. \pageref{problem}).

\begin{figure}[h]\label{problem}

	\includegraphics[width=\textwidth]{problem.pdf}
	\caption{Электрон в электромагнитном поле. Геометрия задачи, начальные условия.}
\end{figure}

\section{Аналитическое решение}
Решение будем искать в нерелятивистском приближении, то есть скорость движения электрона много меньше скорости света. Тогда, согласно второму закону Ньютона уравнение движения электрона имеет вид:
\begin{equation}\label{motion}
m\diff{\mathbf{v}}{t} = \mathbf{F},
\end{equation}
где $m$ - масса электрона, $\mathbf{v}$ - его скорость, $\mathbf{F}$ - действующая сила.

В элетромагнитном поле на заряженные частицы действует сила Лоренца:
\begin{equation}\label{lorenz}
\mathbf{F} = e \mathbf{E} + \frac{e}{c}\left[\mathbf{vB}\right],
\end{equation}
где $e$ - заряд частицы, $с$ - скорость света, $\mathbf{E}$ и $\mathbf{B}$ - напряженности электрического и магнитного поля, соответственно.

Используя уравнения \eqref{motion} и \eqref{lorenz}, с учётом геометрии задачи \ref{problem} запишем следующую систему уравнений.
\begin{equation}\label{v_system}
\begin{cases}
\diff{v_x}{t} = \frac{e}{m}E_x + \frac{e}{cm}\left(v_yB_z\right)	& 	v_x(0) = 10 \\[10pt]
\diff{v_y}{t} = \frac{e}{mc} v_x B_z					&	v_y(0) = 0 \\
\end{cases}
\end{equation}
Продифференцируем первое уравнение по времени
\[
\diff[2]{v_x}{t} =\frac{e}{cm} B_z \diff{v_y}{t},
\]
введём обозначение $\dfrac{e}{cm}B_z = \gamma$ и подставим второе уравнение вместо $\dot{v_y}$. Получим однородное дифференциальное уравнение второго порядка:
\[
\ddot{v_x} - \gamma^2 v_x = 0		
\]
Корни характеристического уравнения равны $\lambda_{1,2} = \pm \gamma$, следовательно общее решение имеет вид:
\begin{equation}\label{vx_common}
v_x = C_1\, e^{\gamma t} + C_2\, e^{-\gamma t}
\end{equation}
Подставим найденное общее решение в уравнение 2 системы \eqref{v_system} и проинтегрируем, следовательно общее решение для $v_y$ имеет вид:
\begin{equation}\label{vy_common}
v_y = C_1 \, e^{\gamma t} - C_2 \, e^{-\gamma t} + C_3 .
\end{equation}
Продифференцируем по времени общее решение для $v_x$ \eqref{vx_common} и приравняем к первому уравнению системы \eqref{v_system}. В правой части подставим $v_y$ из \eqref{vy_common}. Произведя тривиальные преобразования найдём, что $C_3 = -\xi$, где $\xi = cE_xB_z^{-1}$ Далее, используя начальные условия определим константы - $C_1 = 5 + \xi$ и $C_2 = 5 - \xi$. Таким образом, решением системы уравений \eqref{v_system} будет:
\begin{align}\label{analytic}
v_x &= (5 + \xi)e^{\gamma t} + (5 - \xi)e^{-\gamma t} \\
v_y &= (5 + \xi)e^{\gamma t} - (5 - \xi)e^{-\gamma t} -\xi \hspace{0.1 cm} .
\end{align}

\newpage
\section{Численное решение}
Согласно уравнениям \eqref{motion}, \eqref{lorenz} численное решение для координаты частицы и скорости её движения в зависимости от времени будем искать для следующей системы дифференциальных уравнений первого порядка в соответсвии с геометрией задачи \ref{problem}.
\begin{equation}
	\begin{cases}
		\dot{v_x} = \left[eE_x + \dfrac{e}{c}\left(v_yB_z\right)\right]m^{-1}	
											& 	v_x(0) = 10 \\[10pt]
		\dot{v_y} = \dfrac{e}{mc} v_x B_z	&	v_y(0) = 0 \\[10pt]
		\dot{r_y} = v_y						&	r_x(0) = 1 \\[10pt]
		\dot{r_y} = v_y						&	r_y(0) = 1 
	\end{cases}
\end{equation}
Данная система решается с помощью модуля integrate библиотеки scipy для языка программирования python. Код программы приведен в разделе \ref{code}. 

На рисунке \ref{graph_t} представлена траектория частицы от начального положения до момента $t = 1$ мкс. На рисункe \ref{graph_s} представлен график зависимости компонент скоростей частицы от времени. Точечными линиями показано аналитическое решение.  Рисунок \ref{graph_m} показывает ошибку численного решения относительно аналитического в разные моменты времени.

Рисунки построены с помощью библиотеки matplotlib.
\begin{figure}
	\centering
	\includegraphics[width=0.8\textwidth]{plotTrajectory.pdf}
	\caption{Траектория частицы за интервал времени $t = [0, 1\text{e-}3]$ сек.}
	\label{graph_t}
\end{figure}

\begin{figure}
	\centering
	\includegraphics[width=0.8\linewidth]{plotSpeed.pdf}
	\caption{Зависимость компонент скоростей частицы от времени. Сплошные линии - численное решение, прерывистые линии - аналитическое}
	\label{graph_s}
\end{figure}

\begin{figure}
	\centering
	\includegraphics[width=0.8\linewidth]{plotMistake.pdf}
	\caption{График зависимости относительной ошибки численного решения для компонент скоростей.}
	\label{graph_m}
\end{figure}

\section{Заключение}
Поставленная задача выполнена в полном объеме. Приведена постановка задачи и соответствующий рисунок. Аналитическое решение получено в предположении о том, что скорость частицы много меньше скорости света. Численное решение получено с помощью средств для языка python,  а именно библиотек scipy, numpy. Относительная погрешность имеет порядок  $1\text{e-}8$, что является очень хорошим результатом. 

Но, стоит обратить особое внимание на то, что за времена менее одной микросекунды частица разгоняется до околосветовых скоростей, следовательно, требуется учитывать релятивистские эффекты.

\clearpage
\section{Код программы}\label{code}
\singlespace
\lstinputlisting[language=Python, firstline=1, lastline=32]{lab5.py}

\end{document} % Конец текста.